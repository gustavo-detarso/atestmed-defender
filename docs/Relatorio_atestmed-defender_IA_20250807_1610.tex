% Created 2025-08-07 qui 16:15
% Intended LaTeX compiler: xelatex
\documentclass[11pt]{article}
\usepackage{graphicx}
\usepackage{longtable}
\usepackage{wrapfig}
\usepackage{rotating}
\usepackage[normalem]{ulem}
\usepackage{amsmath}
\usepackage{amssymb}
\usepackage{capt-of}
\usepackage{hyperref}
\usepackage[brazil, ]{babel}
\usepackage[utf8]{inputenc}
\usepackage[T1]{fontenc}
\usepackage[left=3cm, right=2cm, top=3cm, bottom=2cm]{geometry}
\sloppy
\hyphenpenalty=50
\tolerance=2000
\usepackage{graphicx}
\usepackage{sectsty}
\usepackage{ulem}
\renewcommand{\ULdepth}{1.8pt}
\author{Gustavo M. Mendes de Tarso}
\date{\today}
\title{}
\hypersetup{
 pdfauthor={Gustavo M. Mendes de Tarso},
 pdftitle={},
 pdfkeywords={},
 pdfsubject={},
 pdfcreator={Emacs 28.2 (Org mode 9.5.5)}, 
 pdflang={Pt_Br}}
\begin{document}

\begin{center}
\includegraphics[width=0.6\textwidth]{/home/gustavodetarso/Documentos/.share/png/logolula.png}
\end{center}

\vspace{-1.8cm}
\begin{center}
\textbf{Ministério da Previdência Social}\\
Secretaria do Regime Geral da Previdência Social\\
Departamento de Perícia Médica Federal
\end{center}

\vspace{-0.3cm}
\hrule

\vspace{-0.3cm}
\begin{center}
\textit{Coordenação-Geral de Assuntos Corporativos e Disseminação de Conhecimento}
\end{center}

\vspace{-0.8cm}
\begin{center}
\textbf{Gustavo Magalhães Mendes de Tarso}
\end{center}

\vspace{1.5cm}

\textbf{RELATÓRIO INSTITUCIONAL – SISTEMA atestmed-defender}

\section{Introdução}
\label{sec:orgbb31495}

O presente relatório visa documentar o desenvolvimento e a implementação de um sistema de automação para a extração e análise de dados do Portal PMF. Este projeto foi idealizado e desenvolvido exclusivamente por Gustavo Magalhães Mendes de Tarso, Coordenador-Geral de Assuntos Corporativos e Disseminação de Conhecimento. O sistema foi criado para enfrentar desafios específicos relacionados à complexidade dos arquivos CSV gerados pelo portal, que dificultavam a filtragem de profissionais com base em indicadores-chave de desempenho (KPIs).

\section{Contexto do Problema}
\label{sec:org3b3a8e8}

O principal desafio enfrentado era a extração de dados do Portal PMF, que eram disponibilizados em arquivos CSV complexos. Esses arquivos dificultavam a filtragem eficiente de profissionais conforme os KPIs estabelecidos, como ICRA, IATD e Score Final. A complexidade dos dados tornava o processo de análise manual demorado e propenso a erros, impactando negativamente a capacidade de gestão e tomada de decisão.

\section{Soluções Desenvolvidas}
\label{sec:orgb9f4a68}

Para resolver o problema, foi desenvolvido um sistema automatizado que realiza a filtragem dos dados, calcula os indicadores e gera relatórios de forma eficiente. O sistema utiliza scripts para processar os dados, calcular os KPIs e apresentar os resultados de maneira clara e objetiva. A automação desses processos não apenas economiza tempo, mas também melhora a precisão das análises.

\section{Indicadores Estratégicos}
\label{sec:org41e8762}

Os KPIs utilizados no sistema são o ICRA (Índice de Conformidade e Risco de Análise), o IATD (Índice de Análise de Tarefas Diárias) e o Score Final. O ICRA avalia a conformidade e o risco associado às análises realizadas, enquanto o IATD mede a eficiência diária dos profissionais. O Score Final é uma combinação dos dois indicadores anteriores, fornecendo uma visão abrangente do desempenho dos profissionais. Esses indicadores são fundamentais para a gestão estratégica, pois permitem identificar áreas de melhoria e alocar recursos de maneira mais eficaz.

\section{Importância para a Gestão Estratégica}
\label{sec:org2ad1370}

O sistema desenvolvido é crucial para a gestão estratégica, pois permite o monitoramento em tempo real dos indicadores de desempenho. Com a capacidade de transformar grandes volumes de dados em informações acionáveis, o sistema apoia a tomada de decisão informada e estratégica. Isso resulta em uma gestão mais eficiente e na melhoria contínua dos serviços prestados.

\section{Importância do Uso de Inteligência Artificial}
\label{sec:org1a6ff0e}

A utilização de inteligência artificial no sistema traz inúmeros benefícios, como a automação das análises de tarefas e a geração de relatórios institucionais. A IA permite processar grandes volumes de dados de forma rápida e precisa, introduzindo inovação na administração pública. Isso não apenas melhora a eficiência operacional, mas também garante que as decisões sejam baseadas em dados concretos e atualizados.

\section{Impactos Institucionais}
\label{sec:orgc7423ce}

O impacto institucional do sistema é significativo, proporcionando um serviço mais justo, célere e de qualidade para a população. Além disso, a automação dos processos resulta em economia para o governo, ao reduzir o tempo e os recursos necessários para a análise de dados. O sistema também contribui para a transparência e a responsabilidade na gestão pública.

\section{Conclusão}
\label{sec:org73020e0}

Em conclusão, o sistema desenvolvido por Gustavo Magalhães Mendes de Tarso representa um avanço significativo na automação e análise de dados do Portal PMF. Ao enfrentar desafios complexos de extração e análise de dados, o sistema melhora a eficiência e a eficácia da gestão pública, beneficiando tanto a administração quanto a população. A implementação bem-sucedida deste sistema destaca a importância da inovação e da tecnologia na modernização dos processos governamentais.
\end{document}